\usepackage[automark]{scrpage2}



\usepackage{makerobust}

%\DeclareRobustCommand{\graffito}[1]{\marginpar{%
%    \slshape\footnotesize%\small%
%    %\ifodd\thepage\raggedright\else\raggedleft\fi%
%    \raggedright
%    \parindent=0pt\lineskip=0pt\lineskiplimit=0pt\baselineskip=11pt
%    \tolerance=2000\hyphenpenalty=300\exhyphenpenalty=300%
%    \doublehyphendemerits=100000\finalhyphendemerits=\doublehyphendemerits%
%    %\raggedright%
%    \hspace{0pt}#1}}

\usepackage[
		%textwidth=15cm,
	    top=4cm,
	    bottom=4cm,
	    left=3.5cm,
	    right=3cm,
%	    includeheadfoot,
%	    showframe,
	    footskip=1cm,
	    marginparwidth=1.9cm,
%	    textheight=22cm
		]{geometry}

\usepackage[english]{babel}
\usepackage[T1]{fontenc}
\usepackage[utf8]{inputenc}
\usepackage{abstract}

%\usepackage{courier}

\usepackage{graphicx}

\usepackage{marvosym}
%\usepackage{array}
%\usepackage{booktabs}

\usepackage[table]{xcolor}
\definecolor{codeback}{gray}{.5}
\definecolor{darkblue}{rgb}{0,0,.5}

\usepackage{listings}
%\usepackage{listingsutf8}
\lstdefinestyle{cmd}{      
      basicstyle=\ttfamily\fontsize{10}{10.6}\selectfont,
      aboveskip={.3\baselineskip},
      %columns=fullflexible,
      showstringspaces=true,
      breaklines=true,
      tabsize=3,
      showspaces=false,
      frame=leftline,
      backgroundcolor=\color{lightgray},
      literate= %
	{Ö}{{\"O}}1 {Ä}{{\"A}}1 {Ü}{{\"U}}1 {ß}{{\ss}}1 {ü}{{\"u}}1 {ö}{{\"o}}1 {ä}{{\"a}}1
}

\lstdefinestyle{cmd_tiny}{      
      basicstyle=\ttfamily\fontsize{7}{9.2}\selectfont,
      aboveskip={.3\baselineskip},
      %columns=fullflexible,
      showstringspaces=true,
      breaklines=true,
      tabsize=3,
      showspaces=false,
      frame=leftline,
      backgroundcolor=\color{lightgray},
      literate= %
	{Ö}{{\"O}}1 {Ä}{{\"A}}1 {Ü}{{\"U}}1 {ß}{{\ss}}1 {ü}{{\"u}}1 {ö}{{\"o}}1 {ä}{{\"a}}1
}

\lstdefinestyle{shell}{
      language=bash,
      identifierstyle=\ttfamily,
      keywordstyle=\bfseries\ttfamily\color[rgb]{0,0,1},
      commentstyle=\color[rgb]{0.133,0.545,0.133},
      stringstyle=\color[rgb]{0,0,1},
      numbers=left, 
      numbersep=7pt,
      numberstyle=\sffamily\tiny,
      basicstyle=\ttfamily\small,
      aboveskip={.3\baselineskip},
      %columns=fullflexible,
      showstringspaces=true,
      breaklines=true,
      tabsize=3,
      showspaces=false,
      frame=leftline,
      backgroundcolor=\color{lightgray},
      literate= %
	{Ö}{{\"O}}1 {Ä}{{\"A}}1 {Ü}{{\"U}}1 {ß}{{\ss}}1 {ü}{{\"u}}1 {ö}{{\"o}}1 {ä}{{\"a}}1
}
\lstdefinestyle{java}{
      language=java,
      identifierstyle=\ttfamily,
      keywordstyle=\bfseries\ttfamily\color[rgb]{0,0,1},
      commentstyle=\color[rgb]{0.133,0.545,0.133},
      stringstyle=\color[rgb]{0,0,1},
      numbers=left,
      numbersep=7pt,
      numberstyle=\sffamily\tiny,
      basicstyle=\ttfamily\small,
      aboveskip={.3\baselineskip},
      %columns=fullflexible,
      showstringspaces=false,
      breaklines=true,
      tabsize=2,
      showspaces=false,
      frame=leftline,
      backgroundcolor=\color{white},
      rulesepcolor=\color{red}
}

\usepackage{framed}
%\usepackage{pstricks}
\usepackage{tikz}
\usetikzlibrary{shapes}
\usetikzlibrary{arrows}
\usetikzlibrary{calc}


\usepackage{calc}
\usepackage{amsmath}
\usepackage{amssymb}
\usepackage{nccmath}
%\usepackage{dsfont}
\usepackage{empheq}
\usepackage[normalem]{ulem}
\newcommand{\msout}[1]{\text{\sout{\ensuremath{#1}}}}
%\usepackage{lmodern}

\setlength{\mathindent}{1ex}

\usepackage{float}


\usepackage{textcomp} %Paket mit Sonderzeichen, Symbolen 
%\usepackage{pifont}% zapfdingbats für besondere Zeichen
\usepackage{eurosym} %schönes Eurozeichen mit \euro auch der Schrift angepaßt
\usepackage{eco}
\usepackage[osf, sc]{mathpazo}
\linespread{1.03}

\usepackage{microtype}
\usepackage[scaled=0.85]{beramono}

%\usepackage[euler-digits]{eulervm}
%\usepackage{fouriernc}

\usepackage[framed]{ntheorem}
\usepackage{footnote}

\usepackage{float}
\usepackage{dblfloatfix}

%\usepackage{cpation}

%\usepackage{multicol}


\PassOptionsToPackage{hyphens}{url}\usepackage[pdftex,breaklinks]{hyperref}


%\usepackage{breakurl}
\hypersetup{%
    colorlinks=true, linktocpage=false, pdfstartpage=1, pdfstartview=FitV,%
    % uncomment the following line if you want to have black links (e.g., for printing)
    colorlinks=false, 
    linktocpage=false, pdfborder={0 0 0}, pdfstartpage=1, pdfstartview=FitV,% 
    breaklinks=true, pdfpagemode=UseNone, pageanchor=true, pdfpagemode=UseOutlines,%
    plainpages=false, bookmarksnumbered, bookmarksopen=true, bookmarksopenlevel=3,%
    hypertexnames=true, pdfhighlight=/O,%hyperfootnotes=true,%nesting=true,%frenchlinks,%
    urlcolor=webbrown, linkcolor=darkblue, citecolor=webgreen, %pagecolor=RoyalBlue,%
    %urlcolor=Black, linkcolor=Black, citecolor=Black, %pagecolor=Black,%
    %pdftitle={\myTitle},%
    %pdfauthor={\textcopyright\ \myName, \myUni, \myFaculty},%
    pdfsubject={},%
    pdfkeywords={},%
    pdfcreator={pdfLaTeX},%
    pdfproducer={LaTeX}%
}





\newcounter{ale}

\newcommand{\abc}{\item[\alph{ale})]\stepcounter{ale}}

\newenvironment{liste}{\begin{itemize}}{\end{itemize}}
\newcommand{\aliste}{\begin{liste} \setcounter{ale}{1}}
\newcommand{\zliste}{\end{liste}}

\newenvironment{abcliste}{\aliste}{\zliste}



\pagestyle{scrheadings}
\clearscrheadfoot
\ofoot{Lab \laboratorynr{} -- \laboratoryname}
\ifoot{\tomas{} \& \franta}
\ihead{\leftmark}
\ohead{\pagemark}


\setheadsepline{0.4pt}
\setfootsepline{0.4pt}


\definecolor{lightgray}{rgb}{.9,.9,.9}
\definecolor{lightblue}{HTML}{CCCCFF}




\clubpenalty = 10000
\widowpenalty = 10000
\displaywidowpenalty = 10000